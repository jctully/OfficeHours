\documentclass[11pt]{article} % The style of the article, and font size
\oddsidemargin -0.0in
\topmargin -0.5in
\usepackage{algorithm}
\usepackage[noend]{algpseudocode}
\usepackage{amsmath}
\textwidth 6.25in
\textheight 9.25in
%\usepackage{algpseudocode} % For creating awesome looking pseudocode
\setlength\parindent{0pt} % Sets the indentation for new paragraphs.
\newlength\myindent
\setlength\myindent{2em}
\newcommand\bindent{%
    \begingroup
    \setlength{\itemindent}{\myindent}
    \addtolength{\algorithmicindent}{\myindent}
}
\newcommand\eindent{\endgroup}
\begin{document}

\title{officeHours\\ CSCI 412, Spring 2019\\ Group Selection Document \\ Version 2.0}
\author{Michael Farrier, Ken Jung, Joseph Tully}
\date{Modified: 03 June 2019}
\maketitle

\section{Introduction}

Western Washington University has no central app, website, or database that contains office hours for every professor in each department on campus. This is because professor's office hours depend on a lot of several factors and that can be hard to keep track of in a central place such as database. However, our proposed solution is to allow professor's to store their office hours in the database and allow someone like a department manager to update these office hours when they request for any changes or when the quarters change. Our app also has a way for professors to choose when they wish to hold office hours using a switch that changes their status (in office or out of office). This is important for professors who have to go out on conferences or sick leave. With the introduction of a student wait list for office hours, professors can also be conscious about how much time they spend with a student, which is a common issue where a professor will spend well over 10 minutes while there is 5+ students outside the door trying to get help. The end goal is to develop an app that can assist students and professors by making having office hours in a public central place and allowing them to be easily updated for the professor.

\section{Assumptions \&\ Dependencies}

\subsection{Assumptions}
Students will primarily use the app to check if a professor is available for office hours or queue up in a wait list for the office. Also, all office hours are publicly available and accurate upon obtaining them, especially when a new quarter starts. For professors, we'll assume that the professor is responsible for changing their office hour times and that they'll remove students on the top of the wait list via the app when they are done helping the current student. \\

\pagebreak 

\subsection{Dependencies}
In terms of dependencies, we'll be utilizing a database. This will implemented using Firebase's Cloud Firestore database. This database will need to store multiple data items such as the office hour times, status of the office, students wait listed, office number, building, department, and email. We will also be utilizing the GPS feature built-in to android phones in order to prevent students from joining the wait list if they are not within a certain radius of the office. For example, if a student is waiting for an office in the Communications Facility and moves to Academic West, they will be unable to join the queue because it is unfair for other students to wait an additional amount of time for that student to traverse to the office when another student is ready to attend hours nearby. 

\section{Scope}

\\
\subsection{Scope of Release}
On release, the app will focus on centralizing the computer science department's office hours and making them the first available office hours that can be viewed and queued up by students. We also included two linguistic professors in order to test searching and filtering professors. Professors will be able to change the status of their office and remove the student at the top of the wait list.

\subsection{Scope of Future Releases}
In the future, the app will be extended to more departments. Eventually, Western should consider utilizing our central database that stores all the professor's office hours and have it easily updated via a website portal. Professors can have more control over what they wish to display for their information window such as pictures, profiles, and links to personal websites. As suggested by Phil Nelson, this app can be generalized and branch out to more universities. We can also charge students a small convenience fee of \$1.\ Instead of allowing professors to remove the student currently at the top of the wait list, we want this student at the top to be notified on their smart phone that the professor is ready to see them.

\section{Features}

\begin{tabular}{ |p{7cm}|p{9cm}|  }
\hline
\multicolumn{2}{|c|}{Feature List} \\
\hline
\bfseries{Feature Name}& \bfseries{Description} \\
\hline
Centralized Database & Database will store everything related to a professor's office/office hours and will contain the student wait list.\\
\hline
Student Wait List & Student will be able to queue up for office hours if a student is currently being helped.\\
\hline
Searching and Filtering & Student's can quickly find their professors by using the search bar and typing in departments or the professor's name.\\
\hline
GPS Feature & This will track the student's current location, if they leave are not within a certain radius of that professor's office they cannot join the queue.\\
\hline
Differentiate between Student, Professor & Students will be able to view all the professors and their office hours without logging in. Professors will need to log-in to a control panel which allows them to change the status of their office and notify students.\\
\hline
View Additional Information & Students can tap on a professor to view a more detailed list of information such as which building a professor is in, the office number, contact information, and more.\\
\hline
Usage Guide & Students and Professors can use the usage guide to find out how to navigate the app.\\
\hline
Recycler View & Students can quickly scroll through professors and tap on professors by going through the recycler view. This is the first screen users will see in order to quickly get to the point of the app.\\
\hline
\end{tabular}

\end{document}
